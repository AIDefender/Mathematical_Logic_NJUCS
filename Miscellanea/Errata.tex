\documentclass{article}
\usepackage{ctex}
\usepackage{fontspec, xunicode, xltxtra}
\usepackage{amssymb}
\usepackage{amsmath}
\usepackage{amsfonts}
\usepackage{mathrsfs}
\usepackage{enumerate}
\usepackage{amsthm}
\usepackage{hyperref}
\usepackage[a4paper,top=30mm,bottom=30mm,left=30mm,right=30mm]{geometry}

\begin{document}
\title{数理逻辑讲义勘误表\footnote{v0.7}}
\author{丁超}
\maketitle	
\begin{enumerate}
\item 书后没有索引
\item 没有公式相等和逻辑符优先级的定义. 
\item P18-13,14 P48-15,19,20 由于各公式可能永真,永假,或不独立, 导致结论不一定成立. 
\item P47-13 $I(~^{-1})$的定义没有加mod; $x_1x_2$ 中间要加 *. 
\item P47-14(1)就是书上例题但又没有给证明. 有同学写作业就直接当它可用了. 
\item P41 出现了$y\equiv x$但书上没有定义$\equiv$. 事实上我认为多余定义它, 用"="就够了. 
\item P41 case 6.1 若$y\equiv x$那么$x$不在$\forall xA$中自由出现, 也就有$(\forall xA)[t/x] = \forall xA$,实际上没有发生替换也就不用这么麻烦证明. 
\item P48-17 第二个$\Gamma$改为$\Gamma_n$. 
\item P49-21 没有解释一个公式(集)有模型是指这个模型满足它. 
\item P55 左上角用了元定理,不是G的规则. 
\item P72 第三行将c换成y没有说明得到的还是证明树
\item P73 定理6.19 $\Gamma$和$\Psi$不在同一个语言内,要说明$\Gamma$的模型如何由$\Psi$的模型得到. 
\item P75 第六讲习题五没说$\Phi$是公式集. 
\item P80定义7.9前一行L-闭项没有定义,我认为是没有变元的项. 
\item P82第二行闭项没有定义,我认为是闭公式. 
\item P87 公式命名不同于出现顺序, 为何不一律以第一个出现的公式为A...
\item P90 (9) missing ``)''. 
\item P95 第一行起至本章末尾$G$改为$G'$ . 
\item 第03讲Slides第27页 有同学反映[*]没有定义. 
\item 第十讲P108 定义10.5约定(1)A(t)表示A(t), 本页倒数第二行前件多个A. 
\item P121 定义11.1(c)加上$B\subseteq E$. 
\item P122 命题11.2定义加上$C\subseteq \mathcal{P}(E)$. 
\item P138 倒数第三行 $\exists s \in S$改成 $\exists t \in S$. 
\item P141第一行多个$\bot$ 倒数第二行$\psi_1 \psi_2$改成$\varphi_1 \varphi_2$. 
\item P136定义12.1 第二行``$\mathfrak{F}$上的关系''改成``$W$上的关系''. 
\item P143图12-7,$w_6$哪来的?模型中没有. 
\item 第十二讲框架类没有定义. 


\item 定义12.4最好结合一下语义不然很难区别路径公式与状态公式. 
\item P154 (h),(i)一样. 
\item P147 (P1) 第二个$x$改成$s_0$. 
\item 闭公式closed formula是指没有自由变元的公式. 书上可能没有定义. 
\item P140 ``合式公式(well-formed-formula)''没有定义. 
\item P148 图12-9, black与grey颜色相近. 
\item P149 定义12.13 ``K-证明是一个无穷的'' 改成 ``--有穷的''. 
\item P150 例12.9第4步前面少加了$\Box$. 每行前不要加$\vdash$. 

\item Hauptsatz和紧性定理的证明过程完全不看也不影响做习题. 

\item $\leftrightarrow$没有定义. 
%The followings come from Wu Jun. 
\item (开始由吴骏给出)\\P95 倒数第6行$(T32)$改为$(T22)$. 
\item P85 第一行''Sklolem"改为''Skolem''. 
\item P102定理9.7要除去条件$x\notin FV(B)$并重新证明. 
\item P84第4行``$f(t_1)$"后面加一个``)''. 
\item P69定义6.10(1)(2)删除开头的两个``若''. 
\item P12 第4午$A_n$改成$A_m$. 
\item P40 情况1 下一行$M\models_\rho$改为$M\models_\sigma$. 
\item P40 倒数第3, 4行$\frac{t_1}{t_n}$改为$t_1$. 
\item P51 规则$\exists R$上矢列后件在$A[t/x]$前少了$\Lambda, $. 
\item 倒数第三行两处``$\mathscr{L}=$''改成``$\mathscr{L}-$''. \\
		(结束由吴骏给出)
%end of Wu Jun

%By Sun Xudong
\item (由孙旭东指出)P68命题6.8没有说明公式集可满足的定义. 是每个公式可以有不同的模型还是要有统一模型. 虽然从证明中可以看出默认的是后者. 

%By Shen Mingjie
\item (开始由沈明杰指出)\\P72 (5)第二行, ``由前命题''改成``由命题6.5''; ``只需证若$Con(\Psi_n$''改为``--$Con(\Psi$; (6)$\exists x.A \in \Gamma$改为``$\in \Psi$''. 
\item P118 induction
\item P84 第四行``$f(t_1)$''后加一个``)''\\(结束由沈明杰指出)

%By Chen Qinlin
\item (由陈钦霖指出)P118 引理10.18, 归纳步骤,改为``不妨设P终于度为d的Cut推理''. 

\end{enumerate}

\end{document}
