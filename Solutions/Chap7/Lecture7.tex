\documentclass{article}
\usepackage{ctex}
\usepackage{fontspec, xunicode, xltxtra}
\usepackage{amsmath}
\usepackage{amssymb}
\usepackage{mathrsfs}
\usepackage{enumerate}
\usepackage{amsthm}
\usepackage{bussproofs}
\usepackage{hyperref}
%\usepackage{mathabx}
\usepackage[a4paper,top=30mm,bottom=30mm,left=30mm,right=30mm]{geometry}

\newcommand{\vv}[0]{\vdash\dashv}
\renewcommand{\today}{\number\year 年 \number\month 月 \number\day 日}

\setlength\parindent{0pt}
\vspace{-8ex}
%\date{}
\begin{document}

\title{数理逻辑第七讲习题参考答案\footnote{Ver. 1.0. 原答案由宋方敏教授给出手稿, 乔羽同学录入, 最后由丁超同学修订补充. 此文档来源为{\url{https://github.com/sleepycoke/Mathematical_Logic_NJUCS}},
 欢迎各位同学提出意见共同维护. 
}}
\maketitle
\section{}
此题答案由陈邵源给出. 	
\begin{align*}
  &(\forall x \exists y \forall z \exists u P(x, y, z, u))^s \\
  =& \forall x (\exists y \forall z \exists u P(x, y, z, u))^s \\
  =& \forall x (\forall z \exists u P(x, f(x), z, u))^s\quad \text{这里$f$为一元新函数}\\
  =& \forall x \forall z (\exists u P(x, f(x), z, u))^s \\
  =& \forall x \forall z (P(x, f(x), z, g(x, z)))^s \quad \text{这里$g$为二元新函数}\\
  =& \forall x \forall z P(x, f(x), z, g(x, z))
\end{align*}

\section{}
此题答案由杨嘉文板书补充得.
\begin{proof}
引入记号``$\vv$'':
$$A \vv B \Leftrightarrow A\vdash B \text{且} B \vdash A. $$
那么显然有
$$\vdash A \leftrightarrow B \Leftrightarrow A\vv B $$
且它是一个等价关系. 
\begin{align*}
  &(\forall x P(x) \wedge \forall y Q(y)) \rightarrow \exists z P(z) \\
  \vv & (\forall x \forall y (P(x) \wedge Q(y))) \rightarrow \exists z P(z) \quad &\text{命题7.4(3)}\\
  \vv & \exists x \exists y (P(x) \wedge Q(y) \rightarrow \exists z P(z)) \quad &\text{命题7.4(5), 命题7.3(1)}\\
  \vv & \exists x \exists y \exists z (P(x) \wedge Q(y) \rightarrow P(z))\quad &\text{命题7.4(8), 命题7.3(1)}\\
\end{align*}
于是
$$\vdash [(\forall x P(x) \wedge \forall y Q(y)) \rightarrow \exists z P(z)]\leftrightarrow [\exists x \exists y \exists z (P(x) \wedge Q(y) \rightarrow P(z))].$$
\end{proof}


\section{}
此答案由孙旭东的板书精简得.
\begin{proof}
设A中量词的个数为$n$,下面对$n$作归纳证明.
\begin{itemize}
	\item[Basis:] $n = 0$. 此时$A^s = A$, 从而$FV(A^s) = FV(A) = \emptyset$. 
	\item[I.H.:] $n \le k$时$FV(A^s) = \emptyset$. 

	\item[I.S.:] 设$n = k + 1$, $A$呈形$Q x B$, 从而$B$的量词个数为$k$, 且$FV(B)\subseteq FV(A)$. 又$FV(A) = \emptyset$, 则$FV(B) = \emptyset$. 由归纳假设有$FV(B^s) = FV(B) = \emptyset$. 
	
		若$Q$为$\forall$, 那么$A^s = \forall x B^s$.  而又由于$FV(A^s) = FV(B^s) \backslash \{x\}$, $FV(A^s) = \emptyset$. 
		
		若$Q$为$\exists$, 那么由$FV(A)= \emptyset$有$A^s = (B[\frac{c}{x}])^s$, $c$为新常元. 
		注意到$B[\frac{c}{x}]$的量词个数不超过$k$且$FV(B[\frac{c}{x}]) = FV(B) \backslash \{x\} = \emptyset $, 那么由归纳假设$FV((B[\frac{c}{x}]^s) = \emptyset$. 
		最后$FV(A^s) = FV((B[\frac{c}{x}])^s) = FV(B^s) =\emptyset$.  
\end{itemize}
综上, $FV(A^s) = \emptyset$. 
\end{proof}

\section{}
先给出一个语义证明:

\begin{proof}
	
设$\mathfrak{M}$为任意结构.

设 $\mathfrak{M} \vDash \exists x \forall y P(x,y)$,以下证明$\mathfrak{M} \vDash \forall y \exists x  P(x,y)$. 

$\mathfrak{M} \vDash \exists x \forall y P(x,y)$

$\Rightarrow$存在$a \in M$ 对任何$b\in M$, 有$(a,b)\in P_M$

$ \Rightarrow$对任何$b\in M$, 有$a\in M$使得$(a,b)\in P_{M}$

$\Rightarrow \mathfrak{M}\vDash \forall y \exists x  P(x,y)$
\end{proof}

再给出陈邵源在$G$中的语法证明:
\begin{proof}
~
  \begin{prooftree}
      \AxiomC{Axiom}
      \noLine
      \UnaryInfC{$P(u, v) \vdash P(u, v)$} \RightLabel{$\forall L, \exists R$}
      \doubleLine
      \UnaryInfC{$\forall y P(u, y) \vdash \exists x P(x, v)$} \RightLabel{$\forall R, \exists L$}
      \doubleLine
      \UnaryInfC{$\exists x \forall y P(x, y) \vdash \forall y \exists x P(x, y)$} \RightLabel{$\rightarrow R$}
      \UnaryInfC{$\vdash \exists x \forall y P(x, y) \rightarrow \forall y \exists x P(x, y)$}
  \end{prooftree}
\end{proof}

\section{}
直接引用陈邵源的答案: 
\begin{proof}
构造模型$\mathfrak{M} = (M, \sigma)$,其中$M = \{1, 2\}$,$P_M = \{(1,1), (2,2)\}$,则易验证$\mathfrak{M} \nvDash_\sigma \forall x \exists y P(x, y) \rightarrow \exists y \forall x P(x, y)$,从而$\nvDash\forall x \exists y P(x, y) \rightarrow \exists y \forall x P(x, y)$. 
\end{proof}


\section{}
先给出一个语义证明:
\begin{proof}
设$\mathfrak{M}$为任意结构

$\mathfrak{M} \vDash \forall x P(x,f(x)) $

$\Rightarrow$对任何$a\in M, (a, f_M(a))\in P_M$

$\Rightarrow$对任何$a\in M $有$b \in M$使$(a, b)\in P_M$

$\Rightarrow \mathfrak{M} \vDash \forall x \exists y  P(x, y)$

所以$\mathfrak{M} \vDash \forall x P(x, f(x)) \Rightarrow \mathfrak{M} \vDash \forall x \exists y  P(x, y)$

即$\vDash \forall x P(x, f(x)) \rightarrow \forall x \exists y  P(x, y)$
\end{proof}

再引用陈邵源在$G$中的语法证明:
\begin{proof}
	~
  \begin{prooftree}
      \AxiomC{Axiom}
      \noLine
      \UnaryInfC{$P(z, f(z)) \vdash P(z, f(z))$} \RightLabel{$\exists R$}
      \UnaryInfC{$P(z, f(z)) \vdash \exists y P(z, y)$} \RightLabel{$\forall L$}
      \UnaryInfC{$\forall x P(x, f(x)) \vdash \exists y P(z, y)$} \RightLabel{$\forall R$}
      \UnaryInfC{$\forall x P(x, f(x)) \vdash \forall x  \exists y P(x, y)$} \RightLabel{$\rightarrow R$}
      \UnaryInfC{$\vdash \forall x P(x, f(x)) \rightarrow \forall x \exists y P(x, y)$}
  \end{prooftree}
\end{proof}

\section{}
此题实际上是定理7.7的一个直接结果. 

首先给出宋方敏教授的证明:
\begin{proof}
$ \forall x \exists y  P(x, y) $可满足

$\Rightarrow$存在结构$\mathfrak{M} \models \forall x \exists y P(x, y)$. 设其论域为$M$. 

$\Rightarrow$对任何$a \in M$有$b \in M$使得$(a, b) \in P_M$. 

$\Rightarrow$对任何$a\in M$, $M_a := \{b|(a, b)\in P_M\} \neq \emptyset$.

由选择公理知,存在$\tau : \mathcal{P} (M) \backslash \{\emptyset\} \rightarrow M$. 



$\Rightarrow$对任何$a\in M$, $\tau(M_a) \in M_a$ 

$\Rightarrow$对任何$a\in M$, $(a,\tau(M_a)) \in P_M$, 

$\Rightarrow$令$f_{M'}=\{(a,\tau(M_a))|a \in M\}$, $\mathfrak{M'} = \mathfrak{M} +f_{M'}$.

那么有$\mathfrak{M'} \vDash \forall x P(x, f(x))$,
即$\forall x P(x, f(x))$可满足. 

\end{proof}

下面给出陈邵源的一个更具体的证明: 
\begin{proof}
	构造模型$\mathfrak{M} = (M, \sigma)$, 其中$M = \{0\}$,$P_M = \{(0, 0)\}$,$f_{M'}(0) = 0$, $\mathfrak{M'} = \mathfrak{M} + f_{M'}$. 则易验证$\mathfrak{M} \models \forall x \exists y P(x, y)$和$\mathfrak{M'} \models \forall x P(x, f(x))$. 从而$\forall x \exists y P(x, y)$可满足$\Rightarrow$ $\forall x P(x, f(x))$可满足. 
\end{proof}

读到这里想必你已经发现,由于此题的两个公式给得太具体了,我们只要证出以下命题之一就足够

\begin{itemize}
	\item $\forall x \exists y P(x, y)$矛盾; 
	\item $\forall x P(x, f(x))$可满足. 
\end{itemize}

而事实上后者很容易就可以构造出模型. 

\section{}

设$A$为$P(f(c))$,

$H_0=\{c\}$,

$H_1= H_0 \cup \{f(t)|t\in H_0\}= \{c\}\cup \{f(c)\} =\{c, f(c)\}$,

$H_2=\{c,f(c),f^2(c)\}$,

...

$H_n=\{c,f(c),...,f^n(c)\}$,

$H=\{f^n(c)|n \in N\}$.


\section{}
丁超认为此题结论要修正为$|H_n| < \aleph_0$而且$|H_A| \le \aleph_0$. 
\begin{proof}
若$A$中无函数符,那么对任何$n$, $H_A = H_n = H_0$, 结论成立. 
	
若$A$中有函数符, 对$n$归纳证明$H_n$有穷. 
\begin{itemize}
\item[Basis:]$n=0, H_0=\{c_0\}$或$\{c|c$为$\alpha$中常元$\}$, 易见$H_n$有穷. 

\item[I.H.:] $H_k$有穷

\item[I.S.:] $H_{k+1} = H_k \cup \{ f(t_1,...,t_m)|t_i \in H_k$, $f$为$A$中$m$元函数符.\}

注意到对于特定的函数符$f_0$,$|\{f_0(t_1,...,t_m)|t_i \in H_k\}| = |H_k^m|$. 
由归纳假设$H_k$有穷,从而上式$< \aleph_0$. 又由于$A$的函数符集是有穷的,$\{ f(t_1,...,t_m)|t_i \in H_k$, $f$为$A$中$m$元函数符.\}有穷. 

从而$H_{k+1}$有穷. 


\end{itemize}
于是$H_n$有穷. 

最后$H_A$是可数个有穷集的并,一定可数. 而由第8题知$H_A$又是无穷的, 那么就一定有$|H_A| = \aleph_0$. 

\end{proof}


\end{document}