\documentclass{article}
\usepackage{amsmath}
\usepackage{amsthm}
\usepackage{hyperref}
\usepackage{amssymb}
\author{DING Chao}

\title{Selected Answers to Exercise 3\footnote{ver 2.0. This document is maintained on \url{https://github.com/sleepycoke/Mathematical_Logic_NJUCS}}}
\begin{document}
\maketitle
\section{P47-13(3)}
\begin{align*}
	 &((x_1*x_2)^{-1})_{M[\sigma]}\\
	=&I(^{-1})((x_1*x_2)_{M[\sigma]})\\
	=&(n - (x_1*x_2)_{M[\sigma]})) \mbox{ mod } n\\
	=&(n - I(*)(x_{1_{M[\sigma]}}, x_{2_{M[\sigma]}})\text{ mod } n\\
	=&(n - (x_{1_{M[\sigma]}} +_n x_{2_{M[\sigma]}})) \mbox{ mod } n\\
	=&(n - (\sigma(x_1) +_n \sigma(x_2))  \mbox{ mod }  n\\
	=&(n - (1  \text{ mod }  n) +_n (2  \text{ mod }   n) )  \mbox{ mod }  n\\
	=&-3  \mbox{ mod }  n
\end{align*}

\bigbreak

\begin{align*}
&((x_1)^{-1}*(x_2)^{-1})_{M[\sigma]}\\
=&I(*)[(x_1^{-1})_{M[\sigma]}, (x_2^{-1})_{M[\sigma]}]\\
=&I(*)[I(^{-1})(\sigma(x_1)), I(^{-1})(\sigma(x_2))]\\
=&((n - 1) \mbox{ mod } n) +_n ((n - 2) \mbox{ mod } n)\\
=&-3 \mbox{ mod } n
\end{align*}

Since $((x_1*x_2)^{-1})_{M[\sigma]} = ((x_1)^{-1}*(x_2)^{-1})_{M[\sigma]}$, by Definition 3.12(2), p36, $A_{M[\sigma]} = T$. 

\section{P47-14(2)}
Lemma 1: $\models A \Rightarrow \models \forall x. A$. ($\Leftrightarrow$ indeed).

\begin{proof}
	By $\models A$, we have for any $(M, \sigma)$, $A_{M[\sigma]} = T$. Therefore for any $a \in M, A_{M[\sigma[x:=a]]} = T$(note that $\sigma[x:=a]$ is one specific assignment of all the $\sigma\,$s), 	which implies $\forall x.A _{M[\sigma]} = T$, aka $M \models _\sigma \forall x.A$ . As a result, since $(M, \sigma)$ is arbitrary, $\models \forall x. A$. 
\end{proof}

Let $A \triangleq (x \doteq y \rightarrow y \doteq x)$, 
\begin{align*}
	 &A_{M[\sigma]}\\
	=&B_\rightarrow[(x \doteq y)_{M[\sigma]}, (y \doteq x) _{M[\sigma]}]\\
	=&\begin{cases}
		F \quad \mbox{ if } (x \doteq y)_{M[\sigma]} = T \mbox{ and } (y \doteq x)_{M[\sigma]} = F\\
		T \quad \mbox{otherwise}%\quad \mbox{ if } (x \doteq y)_{M[\sigma]} = F \mbox{ or } (x \doteq y)_{M[\sigma]} = (y \doteq x) _{M[\sigma]} = T\\
	 \end{cases}\\
\end{align*}

However, $(x\doteq y)_{M[\sigma]} = T$ implies $\sigma(x) = \sigma(y)$, which implies $\sigma(y) = \sigma(x)$ and $(y \doteq x)_{M[\sigma]} = T$. Thus $A _{M[\sigma]} = T$. Applying Lemma 1 twice we get $$\models \forall x \forall y. A.$$ 

\section{P49-24}
\subsection*{$\Rightarrow$}
\begin{align*}
	& M\models _\sigma \forall x. A\\
	\Rightarrow & \text{For all }a \in M, A_{M[\sigma[x:=a]]} = T\\
	\Rightarrow & \text{For all }a \in M, A_{M[\rho[x:=a]]} = T, \text{ since $z$ is fresh in $A$, } \text{ where $\rho \triangleq \sigma[z:=a]$. } \\
	\Rightarrow & \text{For all }a \in M, A_{M[\rho[x := z_{M[\rho]}]]} = T\\
	\Rightarrow & \text{For all }a \in M, A[\frac{z}{x}]_{M[\rho]} = T \text{, by Lemma 3.24, p40. }  \\
	\Rightarrow &M\models _\rho\forall z. A[\frac{z}{x}] \\
	\Rightarrow &M\models _\sigma\forall z. A[\frac{z}{x}] \text{, since $z \notin FV(\forall z. A[\frac{z}{x}])$. }
\end{align*}
\subsection*{$\Leftarrow$}
\begin{align*}
	 &M\models _\sigma\forall z. A[\frac{z}{x}]\\
	\Rightarrow &M\models _\rho\forall z. A[\frac{z}{x}] \text{, since $z \notin FV(\forall z. A[\frac{z}{x}])$, }\text{ where $\rho \triangleq \sigma[z:=a]$. } \\
	\Rightarrow & \text{For all }a \in M, A[\frac{z}{x}]_{M[\rho]} = T \\
	\Rightarrow & \text{For all }a \in M, A_{M[\rho[x := z_{M[\rho]}]]} = T, \text{ by Lemma 3.24, p40. } \\
	\Rightarrow & \text{For all }a \in M, A_{M[\rho[x:=a]]} = T \\
	\Rightarrow & \text{For all }a \in M, A_{M[\sigma[x:=a]]} = T, \text{ since $z$ is fresh in $A$. } \\
	\Rightarrow & M\models _\sigma \forall x. A
\end{align*}


\end{document}