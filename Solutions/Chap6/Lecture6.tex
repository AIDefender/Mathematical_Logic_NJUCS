\documentclass{article}
\usepackage{CJKutf8}%导入CJKutf8包,并且激活中文、日文、韩文的uft8编码

\begin{document}

\begin{CJK*}{UTF8}{gbsn}

\section{}

\subsection{}

反设 Incon($\Phi$),则有在有公式集$S \subseteq \Phi \cap \Psi$ 使 $S \vdash$ 可证, 

因为 $S \subseteq \Phi $ 所以$In con(\Phi)$ 与 $con (\Phi)$ 矛盾。


\subsection{}
情况a,当$\Phi = \Psi$ 时, 1.2 成立。

情况b,当$\Phi = \{ p(a)\}$, $\Phi = \{ \urcorner p(a) \}$

这里 $P $为一元谓词, $a$为常元,易见$con ({\Phi})$,且$con ({\Psi})$,但$Incon (\Psi \cup \Phi)$.
\section{}
\subsection{}
因为 $A , A \rightarrow B \vdash$, 可证,又$A,A\rightarrow B \in \Phi$

所以 由命题9.7知$B \in \Phi$

\subsection{}
因为 $\forall x. A \vdash A [\frac{t}{x}]$ 可证

所以 由命题9.7克制$A [\frac{t}{x}]\in \Phi$

\section{}
设 $\Phi$ 为$\mathcal{L}$的公式集且$con(\Phi)$,

令$\mathcal{L}$ 的全体公式集为$\{\varphi_n | n \in N\}$.

令$\Gamma_0 = \Phi$

\begin{equation*}
\Gamma_{n+1} = \left\{ {\begin{array}{*{20}{c}}
{\Gamma_n \cup \{\varphi_n\}}&{con (\Gamma\cup \{\varphi_n\})}\\
\Gamma_n&{con ( \urcorner \Gamma\cup \{\varphi_n\})}
\end{array}} \right.
\end{equation*}

$\Gamma = \cup_{n\in N} \Gamma_n$, 以下证明

(1)$\Phi \in \Gamma$, 即$\Gamma$为$\Phi$的扩展,易见,

(2)$con(\Gamma)$,对n 作归纳易见$con(\Gamma_n)$

而$\Gamma_0 \subseteq \Gamma_1 \subseteq ... \subseteq \Gamma_n \subseteq ...$ 因此 $con(\Gamma)$.

(3)$\Gamma$为极大协调,即若$con(\Gamma,\varphi_n)$,则$\varphi \in \Gamma$ 

设$con(\Gamma,\varphi_n)$。

case1. $con(\Gamma_n,\varphi_n)$,从而$\varphi_n \in \Gamma_{n+1}$,故$\varphi_n \in \Gamma$

case2. $\urcorner con(\Gamma_n,\varphi_n)$ 从而与$con(\Gamma,\varphi_n)$矛盾,此情况不成立。

因此$\Gamma$ 为$\Phi$ 的扩张且$\Gamma$ 是极大协调的。

\section{}

\subsection{}
令 $P(x)$ 为$x \doteq S$

从而$S \doteq t , P(s) \vdash P(t)$可证,事实上为公理。

即$S \doteq t , S\doteq S \vdash t\doteq S$可证

又$\vdash S\doteq S$可证 

故$S\doteq t \vdash t\doteq S$可证

因此 $\vdash (S\doteq t )\rightarrow (t \doteq s)$ 可证

\subsection{}

令 $P(x)$ 为 $c \doteq u$ ,从而 

$t\doteq S , P(t) \vdash P(S)$ 可证,即

$t\doteq S , t \doteq u \vdash s\doteq u$, 可证 又 $S \doteq t , \vdash t \doteq s$, 可证

故 $S\doteq t , t \doteq u \vdash s\doteq u$ 可证,从而

$\vdash (S \doteq t)\rightarrow ((t \doteq u)\rightarrow(s\doteq u))$可证

\section{}

因为$||\mathcal{L}|| = \aleph_0$

所以$||\mathcal{T}erm|| = \aleph_0$,$||\mathcal{F}ormula|| = \aleph_0$

因为$\Phi{L}$ 又model

$\Rightarrow con(\Phi)$

$\Rightarrow$存在$\Psi \leq \Phi$ 使 $||\Psi|| \leq \aleph_0$ ,且 $\Psi$为 Henkin集。

$\Rightarrow$存在$\Psi \leq \Phi$ 使 $||\Psi|| \leq \aleph_0$ ,且 $\Psi$为 Hintikka集。

$\Rightarrow$存在$\Psi \leq \Phi$ 使 $||\Psi|| \leq \aleph_0$ ,且 $\Psi$有 model其域势$\leq \aleph_0$。

$\Rightarrow$ $\Phi$有model其域势$\leq \aleph_0$。

\section{}
因为 $\Gamma , A\frac{c}{x} \vDash B$

 $\Rightarrow \Gamma , A\frac{c}{x} \vdash B$ 可证

$\Rightarrow \Gamma , A\frac{c}{x} \vdash B$ 有证明树$T(c)$

$\Rightarrow \Gamma , A\frac{y}{x} \vdash B$ 有证明树$T(y)$

这里$T(y)$为在$T(c)$中由$y$势$c$而得$y$为新变元。

$\Rightarrow$ 从而$\Gamma , \exists xA\vdash B$可证

$\Rightarrow \Gamma  , \exists xA\vdash$可证

\section{}

(1),(3),(4)反例

(2)可证

\end{CJK*}

\vspace{0.5cm} % A white space

\noindent

\vspace{0.5cm}

\noindent
\begin{CJK*}{UTF8}{bsmi}
\end{CJK*}

\end{document}
