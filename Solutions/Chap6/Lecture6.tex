\documentclass{article}
\usepackage{CJKutf8}
\usepackage{amsmath}
\usepackage{amssymb}
\usepackage{mathrsfs}
\usepackage{enumerate}
\usepackage{amsthm}
\usepackage{bussproofs}
\usepackage{hyperref}

\renewcommand{\today}{\number\year 年 \number\month 月 \number\day 日}

\setlength\parindent{0pt}
\vspace{-8ex}
%\date{}
\begin{document}

\begin{CJK*}{UTF8}{gbsn}
\title{数理逻辑第六讲习题参考答案\footnote{Ver. 1.0. 原答案由宋方敏教授给出手稿, 乔羽同学录入, 最后由丁超同学修订补充. 此文档来源为\mbox{\url{https://github.com/sleepycoke/Mathematical_Logic_NJUCS}},
 欢迎各位同学提出意见共同维护. 
}}
\maketitle
\section{}

\subsection{}
\begin{proof}[\emph{证明: }]	
设 $\Gamma := \Phi \cap \Psi $.  

反设 $Incon(\Gamma$), 则存在非空有穷公式集 $\Delta \subseteq \Gamma$ 使 $\Delta \vdash$ 可证.  

因为 $\Delta \subseteq \Phi $ 所以 $Incon(\Phi)$, 与 $Con (\Phi)$ 矛盾. 
\end{proof}


\subsection{}
由于等词的存在, 我们可以确保公式集非空. 

设$A$为任一公式,
$\Phi := \{ A\}$, $\Psi := \{  \neg A \}$. 

易见 $Con ({\Phi})$且 $Con ({\Psi})$, 但 $Incon (\Phi \cup \Psi)$.

\section{}
\subsection{}
\begin{proof}[\emph{证明: }]
因为 $A , A \rightarrow B \vdash$, 可证; 

又 $A,A\rightarrow B \in \Phi$, 
由命题6.7知 $B \in \Phi$. 
\end{proof}

\subsection{}
\begin{proof}[\emph{证明: }]

因为 $\forall x. A \vdash A [\frac{t}{x}]$ 可证. 

所以, 由命题6.7可知 $A [\frac{t}{x}]\in \Phi$. 
\end{proof}

\begin{samepage}
\section{}
\begin{proof}[\emph{证明: }]

首先注意根据本书的定义, 一阶语言都是可数的. 那么
设 $\Phi$为$\mathscr{L}$的公式集且$Con(\Phi)$, 
令$\mathscr{L}$的全体公式集为 $\{\varphi_n | n \in N\}$,

$$\Gamma_0 := \Phi$$

\begin{equation*}
\Gamma_{n+1} := \left\{ {\begin{array}{*{20}{c}}
{\Gamma_n \cup \{\varphi_n\}}&{Con (\Gamma\cup \{\varphi_n\})}\\
\Gamma_n&{Incon ( \Gamma\cup \{\varphi_n\})}
\end{array}} \right.
\end{equation*}

$$\Gamma := \cup_{n\in N} \Gamma_n,$$

以下证明
\begin{enumerate}[(1)]

	\item $\Phi \subseteq \Gamma$, 即$\Gamma$为$\Phi$的扩张, 易见. 

	\item $Con(\Gamma)$. 对$n$作归纳易见$Con(\Gamma_n)$. 

若$Incon(\Gamma)$那么存在有穷集$\Delta \subseteq \Gamma$使得$\Delta \vdash$可证. 注意到对于任意$\psi \in \Delta$, 存在$k\in \mathbb{N}$使得$\psi \in \Gamma_k$, 也就是有函数$f : \Delta \rightarrow \mathbb{N}$, $f(\psi) = k$. 取$m := \max\{f(\psi)|\psi \in \Delta\}$, 从而$\Delta \subseteq \Gamma_{m}$. 于是$Incon(\Gamma_{m})$, 矛盾. 


 因此 $Con(\Gamma)$.

	\item $\Gamma$为极大协调. 即若$Con(\Gamma \cup \{\varphi_n\})$,则$\varphi_n \in \Gamma$.

设$Con(\Gamma \cup \{\varphi_n\})$.

Case1. $Con(\Gamma_n \cup \{\varphi_n\})$, 从而$\varphi \in \Gamma_n \cup \{\varphi_n\} =\Gamma_{n+1} $. 又$\Gamma_{n+1} \subseteq \Gamma$, 有$\varphi \in \Gamma$

Case2. $Incon(\Gamma_n \cup \{\varphi_n\})$, 从而有穷集$\Delta \subseteq \Gamma_n$使得$\Delta, \varphi \vdash$可证.又$\Delta \subseteq \Gamma$, 有$Incon(\Gamma \cup \{\varphi_n\})$, 矛盾.
\end{enumerate}

因此$\Gamma$为$\Phi$的扩张且$\Gamma$是极大协调的.
\end{proof}
\end{samepage}

\section{}

\subsection{}
\begin{proof}[\emph{证明: }]

令$P(x)$为$x \doteq s$.

从而$s \doteq t , P(s) \vdash P(t)$可证, 事实上为公理(定义6.10(3)).

即$s \doteq t , s\doteq s \vdash t\doteq s$可证. 

又$\vdash s\doteq s$可证.

故$s\doteq t \vdash t\doteq s$可证(由前两条Cut得). 

因此 $\vdash (s\doteq t )\rightarrow (t \doteq s)$可证(对上条使用$\rightarrow R$). 
\end{proof}

\subsection{}
\begin{proof}[\emph{证明: }]

令 $P(x)$为$c \doteq u$, 

从而
$t\doteq s , P(t) \vdash P(s)$可证, 

即$t\doteq s , t \doteq u \vdash s\doteq u$, 可证. 

又 $s \doteq t \vdash t \doteq s$ 可证,

故 $s\doteq t , t \doteq u \vdash s\doteq u$ 可证(由前两条Cut得), 

从而
$\vdash (s \doteq t)\rightarrow ((t \doteq u)\rightarrow(s\doteq u))$可证(对上条使用两次$\rightarrow R$). 
\end{proof}
\section{}
\begin{proof}[\emph{证明: }]

因为$||\mathscr{L}|| = \aleph_0$(注:事实上按书上定义的一阶语言都是可数的)

所以$||\mathcal{T}erm|| = \aleph_0$, $||\mathcal{F}ormula|| = \aleph_0$. 
令$\mathscr{L'} := \mathscr{L} \cup \{c_n | n \in \mathbb{N}\}$,

$\Phi$\text{有模型}
$\Rightarrow Con(\Phi)$
$\Rightarrow \text{存在}\Phi\text{在}\mathscr{L'}\text{中的Henkin集扩张}\Psi. $\text{(定理6.17)}. 

又$\Psi$是Hintikka集(定理6.18), 可根据定义3.11和引理3.33找到$\Psi$的一个$\mathscr{L'}$中的可数模型$\mathbb{H}$. 由于$\Phi \subseteq \Psi$, $\mathbb{H}$也同时满足了$\Phi$中的所有公式. 那么我们最后只要定义$\mathbb{M}$为$\mathbb{H}$去掉$\mathscr{L'} \backslash \mathscr{L}$中额外加入的常元的解释后得到的结构, $\mathbb{M}$就是$\Phi$的一个(在$\mathscr{L}$中的)可数模型. 
\end{proof}

此问题有更一般的结果,参见 L\"owenheim-Skolem Theorem. 
\section{}
\begin{proof}[\emph{证明: }]

$\Gamma , A[\frac{c}{x}] \models B$

 $\Rightarrow \Gamma , A[\frac{c}{x}] \vdash B$可证(完全性). 

$\Rightarrow \Gamma , A[\frac{c}{x}] \vdash B$ 有证明树, 设为$T(c)$. 

$\Rightarrow \Gamma , A[\frac{y}{x}] \vdash B$ 有证明树$T(y)$. 

这里$T(y)$为在$T(c)$中由$y$替换$c$而得, 且$y$为新变元. 理由同命题10.10. 可由结构归纳得. 

$\Rightarrow\Gamma , \exists xA\vdash B$可证(规则$\exists L$). 

$\Rightarrow \Gamma  , \exists xA\models B$(有效性). 
\end{proof}
\section{}
本题借用了LK中的$WL,WR$规则, 其定义参见定义10.6, 在G中的正确性由命题4.12保证. 
\begin{enumerate}[(1)]
\item 反例: $M:=\{0,1\}$, $P_M:=\{0\}$. 
\item 
		
\begin{proof}[\emph{证明: }] ~
\begin{prooftree}
\AxiomC{$Q(z), P(z)\vdash Q(z)$}
\RightLabel{$\rightarrow R$}

\UnaryInfC{$ Q(z)\vdash P(z)\rightarrow Q(z) $}
\RightLabel{$WL$}
\UnaryInfC{$ \forall y Q(y) , Q(z) \vdash P(z) \rightarrow Q(z)$}
\RightLabel{$\forall L, \forall R$}
\UnaryInfC{$\forall y Q(y) \vdash \forall x (P(x) \rightarrow Q(x))$}

\AxiomC{$P(z)\vdash Q(z), P(z)$}
\RightLabel{$\rightarrow R$}
\UnaryInfC{$\vdash P(z) \rightarrow Q(z), P(z)$}
\RightLabel{$WR$}
\UnaryInfC{$\vdash P(z) \rightarrow Q(z), P(z),\exists x P(x)$}
\RightLabel{$\exists R, \forall R$}
\UnaryInfC{$\vdash \forall x(P(x) \rightarrow Q(x)), \exists x P(x)$}

\RightLabel{$\rightarrow L$}
\BinaryInfC{$\exists x P(x) \rightarrow \forall yQ(y) \vdash \forall x(P(x) \rightarrow Q(x))$}
\RightLabel{$\rightarrow R$}
\UnaryInfC{$\vdash (\exists x P(x) \rightarrow \forall yQ(y)) \rightarrow (\forall x(P(x) \rightarrow Q(x))$}
\end{prooftree}
%这里$z$是新变元. 不必说明
\end{proof}




\item 反例: $M:=\{0,1\}$, $P_M:= Q_M := \{0\}$. 

\item 
\begin{proof}[\emph{证明: }]
设$A$为任一公式, 我们有:
\begin{prooftree}
\AxiomC{$\neg A\vdash \neg A$}
\AxiomC{$A\vdash A$}
\RightLabel{$\neg R$}
\UnaryInfC{$\vdash A, \neg A$}
\RightLabel{$\rightarrow L$}
\BinaryInfC{$A \rightarrow \neg  A \vdash \neg A$}

\AxiomC{$A\vdash A$}
\AxiomC{$A\vdash A$}
\RightLabel{$\neg R$}
\UnaryInfC{$\vdash A, \neg A$}
\RightLabel{$\rightarrow L$}
\BinaryInfC{$\neg A \rightarrow A \vdash A$}
\RightLabel{$\neg L$}
\UnaryInfC{$\neg A \rightarrow A, \neg A \vdash $}


\RightLabel{$Cut, \wedge L$}
\BinaryInfC{$A \leftrightarrow \neg A \vdash$}
\end{prooftree}

从而有$R(z, z)\leftrightarrow \neg R(z, z)\vdash$可证. 再注意到
$$R(z, z) = R(x, z)[\frac{z}{x}], \quad \neg R(z, z) = \neg R(x, x) [\frac{z}{x}]$$
继而有
$$R(z, z)\leftrightarrow \neg R(z, z) = (R(x, z)\leftrightarrow \neg R(x, x))[\frac{z}{x}]. $$
那么我们有证明树
\begin{prooftree}
\AxiomC{$R(z, z)\leftrightarrow \neg R(z, z)\vdash$}
\RightLabel{$WL$}
\UnaryInfC{$\forall x(R(x, z)\leftrightarrow \neg R(x, x)), R(z, z)\leftrightarrow \neg R(z, z)\vdash$}
\RightLabel{$\forall L$}
\UnaryInfC{$\forall x(R(x, z)\leftrightarrow \neg R(x, x))\vdash$}\RightLabel{$\exists L$}
\UnaryInfC{$\exists y\forall x(R(x, y)\leftrightarrow \neg R(x, x))\vdash$}
\RightLabel{$\neg R$}
\UnaryInfC{$\vdash \neg \exists y\forall x(R(x, y)\leftrightarrow \neg R(x, x))$}

\end{prooftree}
\end{proof}
\end{enumerate}


\end{CJK*}


\end{document}