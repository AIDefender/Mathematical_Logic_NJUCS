\documentclass{article}
\usepackage{ctex}
\usepackage{fontspec, xunicode, xltxtra}
\usepackage{amsmath, amsfonts, amssymb}
\usepackage[a4paper,top=30mm,bottom=30mm,left=30mm,right=30mm]{geometry}
\usepackage[shortlabels]{enumitem}
\usepackage{bussproofs}

\setmainfont{Times New Roman}
\newcommand{\combine}[2]{{C_{#1}^{#2}}}

\setlist{nolistsep}

\begin{document}
\title{数理逻辑作业}
\author{陈劭源 \and 161240004}
\maketitle

\section*{3}
\begin{enumerate}[(a)]
  \item $\mathfrak{F} = (W, R)$,其中$W = \{a, b\}$, $R = \{(a, b)\}$,则对于状态$a$,$\square \bot$不成立(无论标记函数是什么),从而$\square \bot$ 在该框架下非有效。

  该公式在满足$R = \varnothing$的框架$\mathfrak{F} = (W, R)$上均有效。

  \item $\mathfrak{F} = (W, R)$,其中$W = \{a, b, c\}$, $R = \{(a, b), (a, c)\}$,则对于标记函数$L = \{(a, \varnothing), (b, \{p\}), (c, \varnothing)\}$,在模型$(\mathfrak{F}, L)$ 下,公式$\lozenge p \rightarrow \square p$ 在状态$a$上非真,从而该公式在上述框架下非有效。

  该公式在满足对于任意$u \in W$,存在至多一个$v \in W$使得$Ruv$的框架$\mathfrak{F} = (W, R)$上均有效。
  \item $\mathfrak{F} = (W, R)$,其中$W = \{a, b\}$, $R = \{(a, b)\}$,则对于标记函数$L = \{(a, \{p\}), (b,\varnothing)\}$,在模型$(\mathfrak{F}, L)$ 下,公式$p \rightarrow \square \lozenge  p$ 在状态$a$上非真,从而该公式在前述框架下非有效。

  该公式在所有对称框架下均有效。
  \item $\mathfrak{F} = (W, R)$,其中$W = \{a, b, c, d\}$, $R = \{(a, b), (b, c), (b, d)\}$,则对标记函数$L = \{(a, \varnothing), (b,\varnothing), $  $(c, \{p\}), (d,\varnothing) \}$,在模型$(\mathfrak{F}, L)$ 下,公式$\square \lozenge p \rightarrow \lozenge \square p$ 在状态$a$上非真,从而该公式在前述框架下非有效。

  该公式在状态转换图呈一个环形的所有框架下均有效。
\end{enumerate}

\section*{4}
\begin{enumerate}[(a)]
  \item[(a)] 1, 2, 3, 4
  \item[(b)] 无
  \item[(c)] 1, 2, 3, 4
  \item[(h)] 1, 2, 3, 4
  \item[(k)] 1\footnote{丁超注: $\forall(y \mathcal{U} r)$ 只在状态1上有效, 因为其它结点都有不经过标记$r$的全路径. }
\end{enumerate}

\section*{6}
\subsection*{$(\square p \wedge \lozenge q) \rightarrow \lozenge (p \wedge q)$的$\mathbf{K}$-证明}
\begin{enumerate}
  \item $\square (p \rightarrow q) \rightarrow (\square p \rightarrow \square q)$ \hfill $\mathbf{K}$
  \item $\square (p \rightarrow \neg q) \rightarrow (\square p \rightarrow \square \neg q)$ \hfill $\mathbf{US}: 1$
  \item $(p \rightarrow q) \rightarrow \neg(p \wedge \neg q)$ \hfill $\mathbf{TAUT}$
  \item $(\square p \rightarrow \square \neg q) \rightarrow \neg(\square p \wedge \neg \square \neg q)$ \hfill $\mathbf{US}: 3$
  \item $\square (p \rightarrow \neg q) \rightarrow \neg(\square p \wedge \neg \square \neg q)$ $\hfill \mathbf{PL}: 2, 3$

  \item $\square(\neg (p \wedge q) \rightarrow (p \rightarrow \neg q)) $ \hfill $\mathbf{N: TAUT}$
  \item $\square(\neg (p \wedge q) \rightarrow (p \rightarrow \neg q)) \rightarrow (\square \neg (p \wedge q)) \rightarrow \square (p \rightarrow \neg q))$ \hfill $\mathbf{US}: 1$
  \item $(\square\neg (p \wedge q)) \rightarrow \square (p \rightarrow \neg q)$ \hfill $\mathbf{MP}: 6, 7$
  \item $(\square\neg (p \wedge q)) \rightarrow \neg(\square p \wedge \neg \square \neg q)$ \hfill $\mathbf{PL}: 8, 5$
  \item $(\square p \wedge \neg \square \neg q) \rightarrow (\neg \square\neg (p \wedge q))$ \hfill $\mathbf{PL}: 9$

  \item $(\square p \wedge \lozenge q) \rightarrow \lozenge (p \wedge q)$ \hfill $\mathbf{PL}: 10, \mathbf{DUAL}, (\mathbf{US: DUAL})$
\end{enumerate}

\subsection*{$\lozenge(p \vee q) \leftrightarrow (\lozenge p \vee \lozenge q)$的$\mathbf{K}$-证明}
\begin{enumerate}
  \item $\square (p \wedge q \rightarrow p)$ \hfill $\mathbf{N: TAUT}$
  \item $\square (p \wedge q \rightarrow p) \rightarrow (\square (p \wedge q) \rightarrow \square p)$ \hfill $\mathbf{US: K}$
  \item $\square (p \wedge q) \rightarrow \square p$ \hfill $\mathbf{MP}: 1, 2$
  \item $\square (p \wedge q) \rightarrow \square q$ \hfill 同理
  \item $\square (p \wedge q) \rightarrow (\square p \wedge \square q)$ \hfill $\mathbf{PL}: 3, 4$

  \item $(\square p \wedge \square q) \rightarrow \square (p \wedge q)$ \hfill 课本例12.9
  \item $(\square p \wedge \square q) \leftrightarrow \square (p \wedge q)$ \hfill $\mathbf{PL}: 5, 6$
  \item $(\square \neg p \wedge \square \neg q) \leftrightarrow \square (\neg p \wedge \neg q)$ \hfill $\mathbf{US}: 7$
  \item $(\neg \square \neg (p \vee q)) \leftrightarrow ((\neg \square \neg p) \vee (\neg \square \neg q))$ \hfill $\mathbf{PL}: 8$
  \item $\lozenge(p \vee q) \leftrightarrow (\lozenge p \vee \lozenge q)$ \hfill $\mathbf{PL}: 9, \mathbf{DUAL}, (\mathbf{US: DUAL})$
\end{enumerate}

\section*{7}
易验证$\mathbf{S4}$对于所有自反且传递的框架均是可靠的(新增加的$p \rightarrow \lozenge p$ 要求框架具有自反性)。为证明$p \rightarrow \square \lozenge p$在$\mathbf{S4}$中不可证,构造模型$\mathfrak{M} = (\mathfrak{F}, L)$,其中$\mathfrak{F} = (\{a, b\}, \{(a,a), (a,b), (b,b)\})$, $L = \{(a, \{p\}), (b, \varnothing)\}$。易验证框架$\mathfrak{F}$自反且传递,但$p \rightarrow \square \lozenge p$在该模型的状态$a$上不为真,从而该公式在$\mathbf{S4}$中不可证。

记$\mathbf{S5}$系统中新增加的公理为:
\begin{quote}
\begin{description}
  \item[$\mathbf{T'}$] $p \rightarrow \lozenge p$
  \item[$\mathbf{4'}$] $\lozenge \lozenge p \rightarrow \lozenge p$
  \item[$\mathbf{B}$] $p \rightarrow \square \lozenge p$
\end{description}
\end{quote}

则$\lozenge \square p \rightarrow \square p$在该系统中的证明为:
\begin{enumerate}
  \item $p \rightarrow \square \lozenge p$ \hfill $\mathbf{B}$
  \item $\lozenge \square p \rightarrow \square \lozenge \lozenge \square p$ \hfill $\mathbf{US}: 1$
  \item $\lozenge \lozenge p \rightarrow \lozenge p$ \hfill $\mathbf{4'}$
  \item $\lozenge \lozenge \square p \rightarrow \lozenge \square p$ \hfill $\mathbf{US}: 3$
  \item $\square(\lozenge \lozenge \square p \rightarrow \lozenge \square p)$ \hfill $\mathbf{N}: 4$

  \item $\square(\lozenge \lozenge \square p \rightarrow \lozenge \square p) \rightarrow (\square \lozenge \lozenge \square p \rightarrow \square \lozenge \square p)$ \hfill $\mathbf{US: K}$
  \item $\square \lozenge \lozenge \square p \rightarrow \square \lozenge \square p$ \hfill $\mathbf{MP}: 5, 6$
  \item $\neg p \rightarrow \square \lozenge \neg p$ \hfill $\mathbf{US}: 1$
  \item $\square \neg \neg p \rightarrow \neg \lozenge \neg p$ \hfill $\mathbf{PL: (US: DUAL)}$
  \item $\square (p \rightarrow \neg \neg p)$ \hfill $\mathbf{N: TAUT}$

  \item $\square (p \rightarrow \neg \neg p) \rightarrow (\square p \rightarrow \square \neg \neg p)$ \hfill $\mathbf{US: K}$
  \item $\square p \rightarrow \square \neg \neg p$ \hfill $\mathbf{MP}: 10, 11$
  \item $\square p \rightarrow \neg \lozenge \neg p$ \hfill $\mathbf{PL}: 12, 9$
  \item $\lozenge \neg p \rightarrow \neg \square p$ \hfill $\mathbf{PL}: 13$
  \item $\square \lozenge \neg p \rightarrow \square \neg \square p$ \hfill $\mathbf{MP:(N}: 14 \mathbf{, US:K)}$

  \item $\neg \square \neg \square p \rightarrow p$ \hfill $\mathbf{PL}: 8, 15$
  \item $\neg \square \neg \square p \leftrightarrow \lozenge \square p$ \hfill $\mathbf{US: DUAL}$
  \item $\lozenge \square p \rightarrow p$ \hfill $\mathbf{PL}: 17, 16$
  \item $\square \lozenge \square p \rightarrow \square p$ \hfill $\mathbf{MP:(N}: 18 \mathbf{, US:K)}$
  \item $\lozenge \square p \rightarrow \square p$ \hfill $\mathbf{PL}: 2, 7, 19$
\end{enumerate}
\end{document}
